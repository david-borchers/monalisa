\documentclass[a4paper,12pt]{article}
\usepackage{graphicx}
\usepackage{subcaption}
\usepackage{lineno}
\usepackage{float}
\usepackage{amsmath}
\usepackage{bm}
 \usepackage[authoryear,round, longnamesfirst]{natbib}
\usepackage{textcomp}
\linenumbers
\linespread{1.6}
\begin{document}
\title{Title}
\author{Authors}
\maketitle

\begin{abstract}

\end{abstract}

\section{Introduction}

Spatial capture-recapture (SCR) models are a set of methods for modelling capture-recapture data obtained from detectors such as camera-traps, hair snag traps and live-capture surveys in which individuals can be identified. Using the location of the detectors (traps) and also the detections of individuals over several occasion (capture histories), a estimate of density of individuals within a population can be obtained as well as an estimate of population size \citep*{Borchers2012}. In order to do this, a spatial model of the population and a spatial model of the detection process are fitted to the capture histories of detected individuals, using either inverse prediction and maximum likelihood \cite{} or data augmentation and Markov chain Monte Carlo methods \cite{}. 

SCR model results are often presented graphically in the form of spatial ``animal density'' maps, these being, on the surface of it, easy to absorb and interpret. However, there are a number of different ways in which ``animal density'' can be interpreted, and the term been used in different contexts in the literature, resulting in substantial confusion around what SCR maps can and cannot do. 

Our main motivation for writing this paper is to correct a misconception that maps of posterior activity center densities can be interpreted as species distribution models (SDMs), as is claimed in \cite[e.g.][]{???}. We argue that this is a mistake because posterior activity center densities chiefly reflect the activity centres of the animals that we have detected on our particular survey. This means that the surface (a) depends heavily on where traps are located, and (b) is very much dependent on a particular realisation of the state process, and does not tell you much at all about the population as a whole. In contrast, all SDMs that we have encountered attempt to establish {\it generally} favourable habitats by estimating relationships between spatially-varying environmental covariates and observed patterns of presence and absence. The area of highest density in a SDM is not necessarily where there are currently most animals (if, for example, animals are {\it not} found in other locations with similar environmental conditions). 

The remainder of the paper illustrates and elaborates on points (a) and (b) above, and suggests a way forward. We argue that confusion around what SCR maps can and cannot do would be minimized by explicitly distinguishing between three kinds of ``animal density'' maps produced by SCR models, each of which conveys useful ecological information:

\begin{enumerate}
\item the intensity of the underlying state process determining where animals are likely to have their activity centers ``on average''  i.e.\ over many realizations of the process. Or, the expected number of activity centres within a region under the state model, divided by the area of the region\footnote{More precisely, the ``expected number of activity centres per unit area'' is the limit of this quantity as the area of the region approaches zero. Similar interpretations apply to the other definitions.}. We call this the {\it expected activity center density}.
\item the number of activity centres within some region, divided by the area of the region, under a single realisation of the state model i.e.\ the current survey. We call this the {\it realised activity center density}.
\item the average number of animals within some region, divided by the area of the region, under a single realisation of the state model. We call this the {\it realised animal density}.
\end{enumerate}

FROM BEN: Give examples of papers where people have misinterpreted the sum of posterior AC distributions under either definition (1) or (3) above. Dorazio uses interpretation (1) because they explicitly talk about the distribution of activity centres. Alexander et al. is a little ambiguous, and could be either definition (1) or (3) depending on whether or not they are talking about density under their particular realisation of the state process, but not (2), because they refer to "density of animals" and not "activity centres". Elliot \& Gopalaswamy (2017) do the same thing as Alexander et al. in their caption of Fig 2.

As a motivating example for the categorisation above, suppose that animals' activity centers have been generated by a point process whose intensity surface is shown in Figure \ref{introplot}a (with intensity increasing eastwards), and with a single realisation of this process overlain (as black dots). In reality the locations of the dots are unknown and must be estimated, but whether the estimated locations of the dots are of interest themselves, or only as a means of estimating the underlying intensity surface, depends on what the goal of the analysis is. If one is interested in where activity centers currently are, one is looking for the dots. If one is interested in which areas are favourable for the location of activity centers {\it in general}, then the dots are only a means with which to estimate the underlying intensity surface. The interpretation of these surfaces differs, but so does their estimation. Expected activity center densities are estimated by modelling the intensity of the state process as a function of environmental covariates, realised activity center densities by summing posterior activity center densities across detected and undetected animals.

Figure \ref{introplot}b imagines that these realised locations were observed with some small amount of measurement error, and plots the sum of the ``posterior distributions'' for each point that account for this uncertainty. This is -- ignoring for now the complexities of fitting an SCR model -- the surface that we argue is incorrectly interpreted as a species distribution model. It is clearly different to the underlying surface shown in Figure \ref{introplot}a. Figure \ref{introplot}c introduces animal movement around the activity centres, representing the expected density of animals at locations at an arbitrary discrete point in time. The estimation of realised animal densities has not been previously described, but also sums posterior densities across animals (we describe this in more detail in Section ???). The surfaces in Figure \ref{introplot}a-c represent the three kinds of animal density surfaces described above -- expected activity center density, realised activity center density, and realised animal density, respectively. A key point we return to in several places in the paper is that these three surfaces are all of potential ecological interest, but that they differ in fundamental ways, and that it is important to keep the distinctions between the surfaces clear.

\begin{figure}[htbp]
\centering
\includegraphics[width=\textwidth]{density-plots.png}
\caption{Simulated activity centers (black dots, left-hand panel) generated from a point process with intensity increasing from left to right (surface in left-hand panel), estimates of those activity centers (central panel), and estimates of animal locations (right-hand panel). These represent, respectively, what we term expected activity center, realised activity center, and realised animal density surfaces.}
\label{introplot}
\end{figure}

The remainder of the paper illustrates the dependency of the realised activity center density on trap location using three example datasets of increasing complexity. The first (Section \ref{1dbinom}) simulates a fixed number of animals from a one-dimensional binomial point process, assumes that animals do not move and also assumes an extremely simple detection process. This example primarily illustrates that flat density surfaces away from traps reflect a lack of knowledge of the locations of unobserved animals, and not that the realised activity center density is constant.

The second example (Section \ref{monalisa}) simulates a more realistic two-dimensional Poisson point process. For visual impact our simulated data is drawn from a density surface model based on one of the most recognisable images in Western culture, the Mona Lisa. This example demonstrates that the realised activity center density is dependent on where traps are located, both far from the traps and, perhaps less obviously, in and around the traps. It also introduces the use of covariates as a means of estimating expected activity center densities, and also incorporates animal movement into realised animal density maps.  

A third and final example (Section \ref{nagarahole}) reanalyses data from a previously published study using SCR to estimate the realised activity center density of tigers in the Nagarahole Tiger Sanctuary \cite{?}. This survey used a dense network of traps; by removing some of these we show the dependency of the density surface on trap locations using real data. A discussion section (Section \ref{discussion}) summarizes our main points, and a final section concludes.

%f we use the introduction to (1) define what a species distribution model is, (2) describe how the posterior AC method works, and (3) cite a few papers that have erroneously interpreted the posterior ACs as a species distribution model. Once we start presenting some results in the following sections, it will hopefully be obvious to the reader why posterior ACs shouldn't be interpreted at species distribution models. At the moment, there is a chance they may not see the 'big picture' until they've reached the end of the paper.


%SCR combines a state model describing the distribution of activity centres in the landscape and an observation model relating the probability of detecting an individual at a particular detector to the distance from a central point in each animal's home range.
%
%Consider a survey in which $K$ traps are placed in a region containing animals having home ranges with fixed centres (also known as activity centres). Once an animal is caught in a trap it remains there until it is released. These traps are checked at regular intervals and marked in such a way that their complete capture history is known and are released.  There are $S$ trapping occasions, trap k is located at Cartesian co-ordinates $x_k$ and the location of traps within the survey are at $x=(x_1, ..., x_k)$. The number of unique animals caught are n. $\bm{X}$ is the animal's location and this is it's activity centre.
%
%Let $\omega_{i}=1$ if animal $i$ was captured on any of the $s$ occasions and $0$ if otherwise. If animal $i$ was captured on trap $k$ on occasion $s$ ($s=1, ...,S$) then $\omega_{is}=1$ and $0$ otherwise. The capture history for animal $i$ is $\omega_{i}=(\omega_{i1}, ..., \omega_{iS})$. Let $p_{ks}(\bm{X};\bm{\theta})$ be the probability that an animal with activity centre at $x$ is caught in trap $k$ on occasion $s$, where $\bm{\theta}$ is the capture probability parameter vector.


%Estimated density surfaces and activity centre distributions are usually presented as images, these being easier to absorb and interpret. In order to present our argument and results in a way that is easy to interpret we also largely present our results as images.

\section{Simulating a 1-D binomial point process} \label{1dbinom}

\subsection{Materials and methods}
We simulate a fixed number of animals distributed across a single dimension according to a linear trend, and then model this data using a binomial point process that incorrectly assumes a uniform distribution of animals across space. Here one can think of, for example, a situation in which true density strongly depends on a covariate that varies in space, but that this covariate is unknown. 

We place a fixed number of points $N$ at random locations on the interval $0<x<R$, with $R=15$. Points are generated according to a probability density that makes them more likely to appear near $x=0$ than $x=15$ i.e.\ $f(x)\propto 1-x/15$ for $0<x<15$ and $f(x)=0$ otherwise.

For simplicity we divide the interval into $R=15$ equal-sized regions, thinking of these as cells in a one-dimensional grid. Traps are placed in $T=5$ cells. We make the simplifying assumptions that animals do not move from the cell they are placed in, and that each trap detects animals perfectly within the cell it occupies but cannot detect animals beyond that cell. 

We simulate with $N=50000$ animals and with different trap configurations.

\subsection{Results}
Figure \ref{binom} shows the estimated realised activity center densities -- the estimated number of activity centers for $N=50000$ animals distributed according a binomial point process with density decreasing linearly with $x$. Under the extremely simplified conditions of this example (no movement of animals, perfect detection within cells), SCR recovers the true number of activity centers in each cell that contained a detector, knows how many animals were {\it not} detected\footnote{Here this is because $N$ is known but if $N$ is unknown it can be estimated. A model assuming a constant density and detecting $n$ animals from a perfect survey of $T/R$ of the study area estimates the total number of animals to be $\hat{N}=n/(T/R)$, implying there are $\hat{N}-n$ animals in the area that were not detected by any trap.}, and distributes the activity centers of these undetected animals evenly across the cells that do not contain a detector\footnote{$\hat{N}-n$ activity centers distributed uniformly between $R-T$ trapless cells gives a mean of $(N-n)/(R-T)$ activity centers per cell, close to the mean of the underlying process $N/R$ when $n\ll N$ and $T\ll R$, as would usually be for wildlife surveys.}. 

\begin{figure}[htbp]
\centering
\includegraphics[width=1\textwidth]{binompp_inf.png}
\caption{Estimated numbers of activity centers obtained from a binomial point process with $N=50000$ simulated animals and density decreasing linearly with $x$, no animal movement, and a step detection function that is perfect within cells and zero otherwise.}
\label{binom}
\end{figure}

\section{Simulating the Mona Lisa as a Poisson point process} \label{monalisa}

\subsection{Materials and methods}
We simulate data from a density surface model based on one of the most recognisable images in Western culture, the Mona Lisa. We created a greyscale version of a region of the original image (Figure \ref{mlinputs}, ``Original'') and downscaled this to $50\times 50$ pixel resolution (Figure \ref{mlinputs}, ``Low Res''). The greyscale value in each cell of the ``Low Res'' gives the true density of activity centers in that cell, with lighter areas indicating higher densities. 

We then used the density surface in the ``Low Res'' image to generate two realisations of points from the underlying process. In the first of these we generated the number of points from a single draw from a Poisson distribution with mean 7500, resulting in 7451 activity centers being generated. These represent the true activity centers in our study region (Figure \ref{mlinputs}, ``Simulated''). This realisation has the advantage of closely reproducing the source image, thereby giving the estimated density surface every opportunity to do the same, but is also much larger than the number of centers that would typically be present in a wildlife survey. We therefore also generated a much smaller second realisation of 84 points (Figure \ref{mlinputs}, ``Simulated small''). This realisation clearly no longer captures the Mona Lisa, but is a more useful aid to understanding some aspects of SCR models.

\begin{figure}[htbp]
\centering
\includegraphics[width=1\textwidth]{mona_inputdata.png}
\caption{Input data for the Mona Lisa simulation study. From a grayscale version of the Mona Lisa (``Original''), we created a downscaled 50x50 pixel version (``Low Res'') that represents the true underlying intensity surface and gives the long-run density of activity centers in each cell. We used this surface to generate a single realization of 7451 activity centers (``Simulated'') and 84 activity centers(``Simulated Small''); in wildlife survey terms this image represents the density of the activity centers of all animals present in the study region.}
\label{mlinputs}
\end{figure}

We conducted various simulated surveys of the population, using different arrays of detectors and also varying the number of detection occasions. Different arrays and detection functions were used for the large and small realisations described above. With a large number of activity centers, we simulated capture histories using a half-normal detection function with $g_0$, the probability of detection at a single detector placed in the centre of the home range, set to 0.5, and $\sigma$, the spatial scale parameter, set to 2. We used four different 3x4 arrays (Figure \ref{mona1low}), with array centers $(19,21)$, $(19,33)$, $(28, 21)$, or $(28, 33)$. All arrays were spaced such that they have length 8 in the $x$-plane and 12 in the $y$-plane, and so have an average spacing of $4=2\sigma$. We then simulated a capture history for either one or 20 occasions on each array.

When using relatively few activity centers, visual interpretation was made easier by increasing the spatial scale parameter, effectively increasing the distance animals travel from the activity centers, and also by increasing the distance between detectors. For these cases, we increased $\sigma$ to 4, holding other detection function parameters at their previous values, and used a 3x4 array centered on $(18,34)$ and with an average spacing of 8 between detectors, double that used previously. We simulated capture histories after one, three, ten, and 20 occasions. 

After simulating these capture histories for these arrays, we then created an estimated realised activity centre surface for each of these simulations. In this scenario we assumed a model with constant density and compared the resulting estimated realised activity center surface to the true population density surface.

The next step was to introduce covariates into our density model and see how this affected the estimated surfaces. We generated covariates by manipulating the ``Low Res'' image to obtain further images using simple image processing operations like blurring and shifting. Covariates are thus all functions of the true densities but the strength of the association between the covariate and true density varies substantially. We generated four covariates: a ``strong'' covariate that uses a Deriche (blur) filter with a small range, effectively smoothing the image locally; a ``moderate'' covariate that uses the same blur filter but with a larger range, increasing the amount of smoothing; a ``weak'' covariate that uses the same degree of blurring as the moderate covariate but in addition shifts the image down and to the right, destroying much of the relationship between covariate and density; and a ``locally strong'' covariate, which uses the strong covariate values in the top right hand part of the image and the weak covariate values in the remainder of the image. These covariates are shown in the first row of plots in Figure \ref{covariates}. For each covariate we estimated a corresponding expected activity center density.

We simulated capture histories and created an estimated realised animal density surface i.e.\ including movement, for each of these simulations and compared them to the true population density surface. All computations were done using the {\it secr} package in R version 3.4.3. 

\subsection{Results}

\subsubsection{Estimated realised activity centre densities with many activity centers}

The same patterns hold in two dimensions under the standard wildlife survey assumptions of Poisson-distributed activity centers (with constant intensity) and detectability inversely related to distance from activity center (Figure \ref{mona1low} and \ref{mona1hi}). A single sampling occasion was sufficient to capture the broad features of the Mona Lisa, but only close to where detectors were located (Figure \ref{mona1low}, first row). Away from the detectors the estimated density quickly reverted to close to the estimated mean intensity of the process. Additional sampling occasions resulted in the density of activity centers close to detectors being estimated in much greater detail, but did not affect the surface away from detectors (Figure \ref{mona1hi}, first row). 

Very different relative and absolute densities were obtained depending on where traps are located, even when estimating density {\it in exactly the same region of the surface and where that region is close to the array} (Figure \ref{mona1low} and \ref{mona1hi}, second row). With a single occasion, density was always estimated to be highest nearest the corner where the trap is located (Figure \ref{mona1low}, second row). This pattern occured because the inset region happened to occur in an area of above average density. If instead it occured in a low density region one would see the opposite pattern -- low density in the corner containing a trap, increasing away from the trap. This was clearly visible when a single sampling occasion was used, because the estimated surface reverted quickly to the mean intensity. Additional sampling allowed fine detail in the density surface to be estimated close to traps, with slower reversion to mean intensity, but there was still very clear disagreement between the density surfaces returned by the different arrays (Figure \ref{mona1hi}, second row).

\begin{figure}[htbp]
\centering
\includegraphics[width=1\textwidth]{many_faces_mona_loweffort.png}
\caption{True activity center densities in this realisation (``Simulated'') compared with realised activity center surfaces estimated using different arrays after a single sampling occasion. High density areas are indicated in yellow, low density areas in blue. Detectors are shown as red crosses. Blue squares show the same 4x4 square centered at $(25,29)$, whose vertices are corner detectors from each of 3x4 arrays. Each plot in the second row shows an enlargement of the blue square in the plot above it. White triangles denote the five cells with the highest estimated densities in each of the second row plots.}
\label{mona1low}
\end{figure}

\begin{figure}[htbp]
\centering
\includegraphics[width=1\textwidth]{many_faces_mona_higheffort.png}
\caption{True activity center densities in this realisation (``Simulated'') compared with activity center surfaces estimated using different arrays after 20 sampling occasions. High density areas are indicated in yellow, low density areas in blue. See the caption to Figure \ref{mona1low} for further annotation details.}
\label{mona1hi}
\end{figure}

\subsubsection{Estimated expected activity center densities with many activity centers}

Introducing covariates into the density models allowed us to recover features of the Mona Lisa across the entire image, not just near where detectors were located, although good estimation of activity center locations depended heavily on the availability of good covariates (Figure \ref{covariates}). With our ``strong'' covariate we recovered all of the broad features of the Mona Lisa, and many of the fine scale features such as eyes, shading of clouds, {\it etc}. With the ``moderate'' covariate we recovered broad scale features but no finer details. With a ``weak'' covariate, the estimated density surface essentially reverted to the mean intensity of the process across the entire region. With a ``locally strong'' covariate -- one that is a good indicator of density in some parts of the study region but poor elsewhere -- the dependency on array location was reintroduced. If the array was located where the covariate was strong, the estimated density surface was accurate in that vicinity. If the array was located where the covariate was weak, then the model estimated no relationship between covariate and density and reverted back to the mean intensity everywhere in the region (Figure \ref{covariates}). 

\begin{figure}[htbp]
\centering
\includegraphics[width=1\textwidth]{mona_covariates.png}
\caption{Expected activity center surfaces estimated using a model with density a function on one of four simulated spatially-varying covariates. Covariates are shown in the first row of plots, and were generating by manipulating the true intensity surface (Figure \ref{inputdata}, ``Low Res'') by blurring and shifting operations (see Section \ref{s:simcapthist} for details). High density areas are indicated in yellow, low density areas in blue. Detectors are shown as red crosses.} 
\label{covariates}
\end{figure}

\subsubsection{Estimated realised activity centre densities with few activity centers}

We observed similar patterns under the more ``wildlife survey appropriate'' condition in which we generated only 85 activity centers across the study region (Figure \ref{peaky}). In this case there is a large difference between the mean intensity surface (the Mona Lisa) and the activity center surface in this realization (85 points), and so it is not surprising that the estimated realised activity centre density surface looks nothing like the Mona Lisa (Figure \ref{peaky}, first row). Nevertheless, a model assuming constant density gives increasingly accurate estimates of the locations of activity centers in the vicinity of detectors as survey effort increases, but very little information is obtained elsewhere, and this does not change with survey effort (Figure \ref{peaky}, first row). This gives the estimated activity center surfaces a characteristic pattern -- the surface becomes increasingly peaked or ``spiky'' around detectors as survey effort increases, but remains flat away from the array. 

\subsubsection{Estimated expected animal densities with few activity centers}

Any covariate model returns a surface that is some multiple of the covariate surface. Whether this is a good approximation of the true mean intensity surface depends on the strength of the covariate and sample size. With a strong covariate and sufficient sampling occasions we recovered the Mona Lisa, but with only a single occasion the direction of the relationship was incorrectly estimated, so that dark areas were predicted as light and light areas as dark (Figure \ref{peaky}, second row). This error was corrected by additional occasions. The same pattern occured with a moderate covariate, but the effect of the weaker covariate is clear in that we did not recover as good an approximation of the Mona Lisa (Figure \ref{peaky}, third row). Additional sampling would not help with this. Note that in both cases the surface we recovered is an approximation of the mean intensity surface. It does not give a good approximation to the locations of the 85 activity centers in this particular realization.

\begin{figure}[htbp]
\centering
\includegraphics[width=1\textwidth]{mona_peaky.png}
\caption{Estimates of realised activity center density surfaces from a constant density model (first row) and expected activity center density surfaces from a model with density depending on ``strong'' or ``moderate'' covariates (second and third rows respectively, see Section \ref{s:simcapthist} for details of how covariates were simulated). 85 activity centers were generated across the entire image, drawn from a Poisson process with intensity given by the ``Low Res'' image in Figure \ref{mona_inputs}. High density areas are indicated in yellow, low density areas in blue. Detectors are shown as red crosses.}
\label{peaky}
\end{figure}

\subsubsection{Estimated realised animal densities when there are few activity centers}

Estimated realised animal density surfaces -- those that incorporate animal movement -- were smoother than the density surface of estimated realised activity centers and also less sensitive to survey effort (Figure \ref{move}). The estimated realised animal density surface adds a movement kernel that is insensitive to survey effort to a realised activity center surface that becomes more peaked as survey effort increases, so this is to be expected. Estimated realised animal density surfaces were not ``just'' smoothed versions of the realised activity center surfaces, however. In our example unobserved animals were estimated to be spending their time on the outskirts of the study region, far away from any detectors (Figure \ref{move}, second row), which is quite different from the homogenous surface we obtained away from detectors when looking only at activity centers (Figure \ref{move}, first row). In contrast, the realised animal density surface for captured animals {\it was} essentially a smoothed version of the realised activity center surface around the detector array, and so very similar in terms of the broad patterns it showed (Figure \ref{move}, third row).

\begin{figure}[htbp]
\centering
\includegraphics[width=1\textwidth]{mona_with_movement.png}
\caption{Estimates of realised activity center density surfaces from a constant density model (first row) and realised animal density surfaces incorporating animal movement for both observed and unobserved animals (second row) and for observed animals only (third row). High density areas are indicated in yellow, low density areas in blue. Detectors are shown as red crosses.}
\label{move}
\end{figure}


\section{Camera-trap survey of tigers in Nagarahole, India}

\subsection{Materials and methods}
We reanalysed data obtained from a camera trap survey of tigers {\it Panthera tigris} living in and around the Nagarahole Tiger Reserve of Karnataka, India, as reported in \cite{Dorazio2017}. A full description of the survey can be found in the original reference. The original survey used a spatial array of 162 motion-activated camera traps, these being placed at 2–3 km intervals throughout the area (Figure \ref{tigernocov}, ``All traps''). 

We reanalyzed this data in a likelihood-based framework, first with a model assuming constant density and with three different trap arrays. The first array was the same one employed in the original study. The second was a subset of traps that excluded a large number traps in the interior of the study region, thus leaving a substantial part of the study area unsurveyed (Figure \ref{tigernocov}, ``Subset \#1''). The third used another subset of traps that excluded eight detectors from each of two interior areas of the survey area in which the original survey showed the density of activity centers to be particularly high (Figure \ref{tigernocov}, ``Subset \#2''). 

We then fitted a number of covariate models in which density was assumed to depend on longitude and latitude. We fitted a variety of linear and smooth functions for each of longitude and latitude; the model selected by the AIC was one including a linear effect of latitude only, and we report results from this model only. Finally, we generated realised animal densities for a constant density model with all traps.

\subsection{Results} 
The same broad patterns were visible in our reanalysis of the Nagarahole tiger survey (Figures \ref{tigernocov} to \ref{tigerspaceuse}). 

\subsubsection{Estimated realised activity center densities}

The full array of traps used in the original Nagarahole study clearly showed three areas of high activity center density in the interior of the study region, along $E\approx 625$ and $N=1324, 1330, 1336$ respectively (Figure \ref{tigernocov}, ``All traps, no cov.''). 

When we reran the survey on a subset of traps that excludes traps in the interior of the study region, high density areas in the interior of the region were replaced by a flat surface indicating a homogenous low density, and the three high density regions described above were not detected  (Figure \ref{tigernocov}, ``Subset \#1, no cov.''). We also observed some regions where estimated density {\it increased} after the removal of the interior traps (see the easternmost detectors in Figure \ref{tigernocov}, ``Subset \#1, no cov.''). This occurs when animals have their activity centers near to, but outside, the area circumscribed by an array -- estimated activity centers then tend to be pulled towards the traps that they are closest to. 

With the second subset of traps, which exclude eight detectors from each of two high density interior areas, the constant density model still recognized that activity centers are located in these areas, but the estimated locations of these activity centers showed a clear shift from what was found in the original survey (Figure \ref{tigernocov}, ``Subset \#2, no cov.''). The estimated location of the northernmost of the two activity centers moved to the south east, while the other activity center moved to the south.

\begin{figure}[htbp]
\centering
\includegraphics[width=1\textwidth]{tiger_surfaces_nocovs.png}
\caption{Estimated realised activity center densities of tigers in Nagarahole Tiger Sanctuary, India, obtained using different camera trap arrays. Plots (a), (b), and (c) show estimated densities; plots (d) and (e) show differences between the estimated densities obtained using using trap subset \#1 and \#2 and those obtained using all traps. Detectors are shown as black crosses.}
\label{tigernocov}
\end{figure}

\subsubsection{Estimated expected activity center densities}

The model with the lowest AIC was one expressing mean intensity as a linear function of latitude. The estimated density surface obtained from this model showed the estimated mean intensity increasing southwards across the region, with mean intensity in the extreme south roughly four times that in the extreme north (Figure \ref{tigercov}, ``All traps, northing''). Estimates of expected activity center density were much less variable than estimates of realised activity center density, and were also less sensitive to changes in the array of traps, provided that the array provided sufficient coverage of the covariate space to estimate the covariate relationship (Figure \ref{tigercov}, ``Subset \#1, northing'' and `Subset \#2, northing''). 

\begin{figure}[htbp]
\centering
\includegraphics[width=1\textwidth]{tiger_surfaces_covs.png}
\caption{Estimated expected activity center density of tigers in Nagarahole Tiger Sanctuary, India, obtained using different camera trap arrays. Plots (a), (b), and (c) show estimated intensities (expected activity center densities); plots (d) and (e) show differences between the estimated intensities obtained using using trap subset \#1 and \#2 and those obtained using all traps. Detectors are shown as black crosses.}
\label{tigercov}
\end{figure}

\subsubsection{Estimated realised animal densities}

The estimated realised animal density surface differed markedly from the realised activity center density surface, with these differences neatly illustrating the different purposes of the two surfaces (Figure \ref{tigerspaceuse}). Activity center densities were highest in those cells where sufficient information had been gathered to precisely identify where a single tiger's activity center was. Adding movement to the surface had the effect of dispersing each area of high (activity center) density across a much wider area, the extent of which depended on the estimated range of movement. The estimated spatial scale parameter for the fitted half-normal detection function we used was $\sigma=1.85$km, so that animals can move a substantial distance from their activity centers, relative to the size of the study area. As a result, animal density was highest in areas in which there were several activity centers in relatively close proximity to one another, even if the location of these activity centers was less precisely known than other activity centers. This occured in areas near the southern and south-western borders of the reserve, as well as in a central location near $N=1340$ (Figure \ref{tigerspaceuse}). In contrast, animal density was low in areas that contained only a single activity center, even if the location of the activity center was precisely known (for example at $N=1330$, $E=624$).

\begin{figure}[htbp]
\centering
\includegraphics[width=0.6\textwidth]{tiger_spaceuse.png}
\caption{Estimated (a) realised activity center density surfaces from a constant density model and (b) realised animal density surfaces for tigers in Nagarahole Tiger Sanctuary, India. High density areas are indicated in blue, low density areas in yellow. Detectors are shown as red crosses.}
\label{tigerspaceuse}
\end{figure}


\section{Discussion} \label{discussion}
The realised activity centre density obtained from an SCR model cannot be interpreted as a species distribution model. Species distribution models predict where species are likely to occur by correlating environmental covariates with species occurrence or species density. Their rationale is to find favourable habitats and predict that animals will be found in similar habitats across the study region. A species distribution model will tend to place higher densities at locations where environmental covariates are most favourable, and spatial variation in the density surface will depend mostly on how environmental covariates change across space.

In contrast, the density on a realised activity center surface is often placed in spikes where the model is most certain that an activity center is located. The shape and location of these spikes depends on where traps are located and also on survey effort. Different arrays produce different results and these results can be improbable, in the sense that high density spikes can occur at locations in unfavourable habitats, if there happen to be activity centers at these locations at the time of the survey. A useful metaphor here is of SCR as a torch shining a light onto the true activity center density surface -- what you see depends on where you shine the torch (trap locations) and how brightly you shine it (survey effort). 

Realised activity center surfaces tend to be flat away from where traps are located. It is crucial to understand that this flatness reflects a lack of knowledge about the density surface, and not the distribution of activity centers in this realisation. This point is clearly stated in standard SCR texts\footnote{``As we more away from `where the data live' (away from the trap array) we see that the density approaches the mean density. This is a property of the estimator as long as the detection function decreases sufficiently rapidly as a function of distance. Relatedly, it is also a property of statistical smoothers such as splines, kernel smoothers, and regression smoothers---predictions tend toward the global mean as the influence of data diminishes'' \citep[p165-166][]{Royle}} but is misinterpreted whenever researchers explicitly or implicitly treat realised activity center densities as maps of the spatial distribution of activity centers across the study area (unless the study area is very densely surveyed), as is done in \cite{Alexander,???}. Another way to see that flatness away from traps reflects uncertainty rather than homogenous density is to plot lower and upper percentiles at each pixel, rather than just the posterior mean -- the differences between these percentiles would be large away from traps and small close to traps. It seems that this is rarely done, or at least reported in the literature; a practice that would be worth changing. 

More important for people actually conducting SCR surveys is that the densities obtained close to traps (and even inside the trap array) {\it also} depend on where traps are located. The inset plots of Figure \ref{mona1low} and \ref{mona1hi} show the same region in space, and this region lies within a $2.5\sigma$ range of all trap arrays, where one would expect to be making inferences about activity centers. We obtained very different density surfaces in this area depending on where traps were located. If one was using SCR to identify areas of high density e.g.\ for conservation purposes, or to locate animals, different areas would be identified depending on which array was used. 

SCR models answer the question ``where is an animal with {\it this} spatial capture history likely to have its activity center?'' The answer is always contingent on where traps are located. Changing the locations of detectors also changes the capture history, and thus the answer to the question of where the activity center is located. This occurs regardless of whether one works in a Bayesian or frequentist framework. Precisely the same is true of the estimated realised activity center density surface, which simply sums estimated activity centers across animals. In this case the question being addressed is ``where are the animals with {\it these spatial capture histories} likely to have {\it their} activity centers?'' The dependency on trap location applies to activity centers estimated for detected animals and for those that were not detected. In the latter case we have limited information and our estimates thus become ``nowhere near where traps are located''. 

None of this precludes realised activity center surfaces from being useful sources of information, but they do need to be interpreted with care. For practical purposes this means always interpreting them with the caveat that they depend on where traps are located. Realised activity centre densities do not give proper answers to questions like ``where are the high- and low-density regions?'' because the highest and lowest points of the surface will always be at or near traps; not because these are high- or low-density regions of space, but because this is where the capture histories make us most certain that animals are, or are not, present. They also cannot answer questions like ``are animals clustered in space?'' or ``is animal density heterogeneous?'' because the realised density surface will always exhibit variability, even if animal densities are truly a realisation of a homogeneous Poisson point process.

When estimating the location of a given activity center, the bias caused by trap locations is lowest if the activity center occurs near the center of a dense array of traps, and is highest if traps are all on one side of the activity center or if detections are only made at a single trap. Thus bias can be reduced by using a design that makes it likely that all activity centers in the study region are surrounded by a network of traps. This will be unachievable for most wildlife surveys, as it requires a large number of traps covering an area beyond the study region, and ideally placed at random {\it [[[note: I say `beyond the region' so that activity centers at the borders are also in the center of some array, but not sure this is correct -- ????]]]} . In summary, it is incorrect to interpret the realised activity center density surface as if it indicated where animals currently have their activity centers. 

There is a way of using SCR so that it can be interpreted as a species distribution model -- by modelling the mean intensity of the underlying process as a function of enviromental covariates. Covariates allow one to see beyond the spatial extent of the array (see Figure \ref{covariates}), provided that the relationship between covariate and response is a good one, and that traps cover a sufficient range of covariate values to estimate that relationship. The resulting surfaces are no longer tied to one particular realisation of the Poisson process, and thus carry less information about where current activity centers are located. Rather, they show the mean intensity of the underlying process assumed to generate activity centers; in other words, the estimated expected density of activity centers across all realisations. Expected densities will be highest where environmental covariates are most favourable (such as further south in Figure \ref{tigercov}). They thus answer the question of ``where are the high- and low-density regions?'' in a way that is consistent with how this question is answered by species distribution models. They predict where we would expect to see activity centers, if we were able to observe multiple independent populations distribute themselves across the study region. 

Using covariate models, and associated model-based inference, is not without issues -- there is a danger of extrapolating the density surface beyond the range of covariates around the traps, and the relationship with density and covariate is assumed to be the same everywhere as it is around the traps. The extent to which the expected activity center surface predicts where animals have their activity center {\it in this realization} depends on the strength of the covariate relationship and on the number of activity centers, each of which is an independent draw from the underlying process. In the Nagarahole survey, for example, there is a relatively weak northing covariate and a small number of activity centers, and the estimated expected activity center density surface provides very little information about the location of current activity centers. 

The concept of an activity center is central to SCR models, but for many applications of SCR it may be more appropriate to consider a distribution of space use, taking into account all locations where an animal may have been present, rather than a distribution over activity center locations only. The detection function estimated as part of an SCR model provides information about how far from its activity center an animal may move. This can be easily integrated with the estimated realised activity center density to give an estimated realised {\it animal} density surface. The resulting surface effectively smooths the realised activity center density surface, with the amount of smoothing determined by the distances that animals move, as given by the detection function. As it is based on activity center locations, the animal density surface also depends on where traps are located and on survey effort. However, it depends less heavily on these factors than the activity center surface because the detection function does not depend on them. In particular, the realised animal density surface quickly stops becoming increasingly ``peaked'' as more survey occasions are added.

Ultimately, the appropriate density surface to use depends on the aims of the researcher. We have argued that the estimated realised activity center density surface should not be used as a species distribution model, because of the strong dependence on trap location and survey effort. But if the goal is to identify the activity centers of {\it some} animals currently in the study region (and it does not matter which ones) then it may well be an efficient way of locating these, particularly at the center of the array. If the goal is to actually {\it find} an animal in the study region, then it is less important where animals have their activity centers and more important to know where they spend their time, and the realised animal density surface is most useful. If the goal is to estimate where animals (not just the ones in the current realisation) are likely to have activity centers, then this is a species distribution question and the expected activity center surface, with intensity a function of covariates, should be used.

\section{Conclusions}
This paper demonstrates that the summed posterior distribution of estimated ranges across animals obtained from SCR -- what we call the realised activity center density surface -- cannot be used as a species distribution model. We illustrated this point in a number of ways, first with a binomial point process, then by using the Mona Lisa to simulate a Poisson point process, and finally using data from a real-world camera trap survey. All these examples returned the consistent message that realised activity center density surfaces differ depending on trap location. This dependency is most obvious at large spatial scales, where moving a trap array is like ``shining a torch'' on a particular part of the study area, but is also present within the region in and around the trap array itself. Our main messages are:
\begin{enumerate}
\item Realised activity center density surfaces cannot be interpreted not SDMs. This is both because the surface makes inferences about one realisation of a spatial point process, whereas SDMs make inferences about the long run average of the process; and because the surface depends systematically on the survey design.
\item The realised activity center density surface typically shows steep peaks and troughs close to the center of arrays, defaulting to close to the mean of the underlying process away from the array. A flat density away from traps reflects a lack of knowledge, and not a homogenous sprawl across which density is constant. We should expect some areas away from traps also to show substantial deviations from the process mean -- it is just that we do not know which areas.
\item An SCR model that models mean activity center density as a function of environmental covariates can be interpreted as a SDM. Here the key difference is that the surface obtained from the covariate model -- what we call an expected activity center surface -- is a statement about the mean intensity of the underlying process, and is independent of array location provided that the environmental covariate space has been sufficiently sampled.
\item Realised activity center densities can be extended into realised animal densities by the addition of animal movement. This is done by distributing the probability mass associated with each possible location of a particular activity center across the entire region in which, conditional of that location being the true one, an animal might be detected. The extent of this region is given by the estimated detection function parameters. 
\end{enumerate}

%"OK everyone, put your traps where you'd like to shine your torch, and you'll get a perfectly good estimate of animal density there using our posterior AC map".

\bibliographystyle{newapa}
\bibliography{/Users/Rachel/OneDrive/Summer2017/ML/refs}

\end{document}
